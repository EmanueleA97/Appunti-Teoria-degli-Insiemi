\documentclass[11pt]{scrartcl}
\usepackage[italian]{babel}
\usepackage[sexy]{evan} %evan.sty

%%%%%%%%%%%%%%%%%%%%%%%%%%%%%%%%%%%%%%%%%%%%%%%%%%%%%%%%%%%%%%%%%%%%%%%%%%%%%%
%
% BOOST SOFTWARE LICENSE - VERSION 1.0 - 17 AUGUST 2003
%
% Copyright (c) 2022 Evan Chen [evan at evanchen.cc]
% https://web.evanchen.cc/ || github.com/vEnhance
%
% Available for download at:
% https://github.com/vEnhance/dotfiles/blob/main/texmf/tex/latex/evan/evan.sty
%
% Permission is hereby granted, free of charge, to any person or organization
% obtaining a copy of the software and accompanying documentation covered by
% this license (the "Software") to use, reproduce, display, distribute,
% execute, and transmit the Software, and to prepare derivative works of the
% Software, and to permit third-parties to whom the Software is furnished to
% do so, all subject to the following:
%
% The copyright notices in the Software and this entire statement, including
% the above license grant, this restriction and the following disclaimer,
% must be included in all copies of the Software, in whole or in part, and
% all derivative works of the Software, unless such copies or derivative
% works are solely in the form of machine-executable object code generated by
% a source language processor.
%
% THE SOFTWARE IS PROVIDED "AS IS", WITHOUT WARRANTY OF ANY KIND, EXPRESS OR
% IMPLIED, INCLUDING BUT NOT LIMITED TO THE WARRANTIES OF MERCHANTABILITY,
% FITNESS FOR A PARTICULAR PURPOSE, TITLE AND NON-INFRINGEMENT. IN NO EVENT
% SHALL THE COPYRIGHT HOLDERS OR ANYONE DISTRIBUTING THE SOFTWARE BE LIABLE
% FOR ANY DAMAGES OR OTHER LIABILITY, WHETHER IN CONTRACT, TORT OR OTHERWISE,
% ARISING FROM, OUT OF OR IN CONNECTION WITH THE SOFTWARE OR THE USE OR OTHER
% DEALINGS IN THE SOFTWARE.
%%%%%%%%%%%%%%%%%%%%%%%%%%%%%%%%%%%%%%%%%%%%%%%%%%%%%%%%%%%%%%%%%%%%%%%%%%%%%%

\begin{document}
\title{Elementi Di Teoria Degli Insiemi}
\subtitle{\large\normalfont\rmfamily\scshape APPUNTI DEL CORSO DI ELEMENTI DI TEORIA DEGLI INSIEMI \\ TENUTO DAL PROF. MARCELLO MAMINO}
\author{Diego Monaco \\ \textnormal{\href{d.monaco2@studenti.unipi.it}{d.monaco2@studenti.unipi.it}} \\ Università di Pisa}
\date{Anno Accademico 2022-23}
\maketitle
\newpage

\tableofcontents
\eject
\newpage

\section*{Premessa}
Queste dispense sono la quasi esatta trascrizione in \LaTeX\,delle dispense del corso di Elementi di teoria degli insiemi, tenuto dal prof. Marcello Mamino nell'anno accademico 2022-23.
\section*{Ringraziamenti}

\mbox{}
\vfill
\begin{wrapfigure}{R}{0.2\textwidth}
	\centering
	\href{https://creativecommons.org/licenses/by-nc/4.0/deed.it}{\includegraphics[width=0.2\textwidth]{licenza.png}}
\end{wrapfigure}

Quest'opera è stata rilasciata con licenza Creative Commons Attribuzione - Condividi allo stesso modo 4.0 Internazionale. Per leggere
una copia della licenza visita il sito web \href{http://creativecommons.org/licenses/by-sa/4.0/deed.it}{\textcolor{blue}{https://creativecommons.org/licenses/by-nc/4.0/deed.it}}.\\

\newpage

\newpage
\section{Prologo nel XIX secolo}
La nascita della teoria degli insiemi è una storia complicata di cui so pochissimo. Però, persone che ne sanno molto più di me hanno sostenuto l'opinione che il problema seguente
abbia avuto un ruolo. Come che sia, è almeno un'introduzione possibile.

\begin{problem}
Data una serie trigonometrica:
\[ S(x) = c_0 + \sum_{i=1}^{+\infty}a_i\sin{(ix)}+b_i\cos{(ix)}
	\]
se, per ogni $x \in \RR$, sappiamo che $S(x)$ converge a 0, possiamo dire che i coefficienti $c_0,a_i,b_i$ sono tutti 0?
\end{problem}

Risolto positivamente da \href{https://it.wikipedia.org/wiki/Georg_Cantor}{\textcolor{purple}{Georg Cantor}} nel 1870.

\begin{definition}
Diciamo che $X \subseteq \RR$ è un \vocab{insieme di unicità} se, per ogni serie trigonometrica:
\[ S(x) = c_0 + \sum_{i=1}^{+\infty}a_i\sin{(ix)}+b_i\cos{(ix)}
	\]
vale la seguente implicazione:
\[ \text{$S(x)$ converge a 0 per tutti gli $x\not\in X$} \implies \text{tutti i coefficienti $c_0,a_i,b_i$ sono nulli}
	\]
\end{definition}

\begin{example}
	Per il risultato di Cantor, $\emptyset$ è di unicità.
\end{example}

\begin{problem}
	Quali sottoinsiemi di $\RR$ sono di unicità?
\end{problem}

\begin{fact}
\label{unicità}
$X \subseteq \RR$ è di unicità se (ma non solo se) ogni funzione continua $f : \RR \longrightarrow \RR$ che soddisfi le ipotesi seguenti è necessariamente lineare\footnote{$f(x) = \alpha x + \beta$.}:
\begin{itemize}
	\item per ogni intervallo aperto $\left]a,b\right[$ con $]a,b[ \cap X = \emptyset$, $f_{|\left]a,b\right[}$ è lineare;
	\item per ogni $x \in \RR$, se $f$ ha derivate destre e sinistre in $x$, allora queste coincidono\footnote{Ovvero $f$ non ha punti angolosi.}.
\end{itemize}
\end{fact}

\begin{example}
	$X = \{\ldots,a_{-2},a_{-1},a_0,a_1,a_2,\ldots\} = \{a_i | i \in \ZZ\}$ con $\ldots < a_{-2} < a_{-1} < a_0 < a_1 < a_2 <\ldots$, $\displaystyle\lim_{i \to +\infty} a_i = +\infty$, $\displaystyle\lim_{i \to -\infty} a_i = -\infty$ ha la 
	proprietà data dal \hyperref[unicità]{Fatto 1.5}, quindi è di unicità.
\end{example}

\begin{notexample}
L'intervallo $[0,1]$ o $\RR$ non hanno la proprietà espressa dall'\hyperref[unicità]{Fatto 1.5}.
\end{notexample}

\begin{notexampleb}
Per l'\vocab{insieme di Cantor} non vale il \hyperref[unicità]{Fatto 1.5}.
\end{notexampleb}

Possiamo costruire l'insieme di Cantor a partire dall'intervallo $C_0 = [0,1]$ nel seguente modo:

\begin{center}
	\begin{figure}[h]
		\centering
		\includegraphics[width=12.5cm]{immagini/cantor.png}
	\end{figure}
\end{center}

ovvero, preso l'intervallo $[0,1]$ possiamo dividerlo in tre parti e rimuovere la parte centrale $\displaystyle\left]\frac 13, \frac 23\right[$, chiamiamo gli intervalli rimanenti $C_1$, possiamo iterare il procedimento sui due segmenti di $C_1$ ed ottenere $C_2,C_3,\ldots$, a questo punto 
definiamo l'insieme di Cantor $C$ come:
\[ C := \bigcap_{i \in \NN}C_i
	\]
Esiste una funzione continua (e crescente) $f : \RR \longrightarrow \RR$ detta \vocab{scala di Cantor} (o \vocab{scala del diavolo}), tale che $f^{\prime}(x) = 0$ per $x \not\in C$ e non è 
derivabile in $x \in C$.

\begin{center}
	\begin{figure}[h]
		\centering
		\includegraphics[width=13.5cm]{immagini/scalacantor.png}
	\end{figure}
\end{center}

tale funzione si costruisce aggiungendo tratti costanti (prima $\displaystyle\frac 12$, poi $\displaystyle\frac 14$, $\displaystyle\frac 34$ e così via, dividendo l'intervallo $[0,1]$ sull'asse delle ordinate in parti uguali) alle parti eliminate sull'intervallo
$[0,1]$ sull'asse delle ascisse per costruire l'insieme di Cantor.

\begin{note}
Per $\QQ$ e $C$ non vale il \hyperref[unicità]{Fatto 1.5} ma, in realtà, sono di unicità.
\end{note}

\begin{exampleb}
L'insieme degli elementi di una successione crescente col suo limite è un esempio di insieme di unicità.
\end{exampleb}

\begin{center}
	\begin{figure}[h]
		\centering
		\includegraphics[width=10.5cm]{immagini/succunic.png}
	\end{figure}
\end{center}

Dimostriamo quindi che $X$ è un insieme di unicità.
\begin{proof}
La funzione $f$ è lineare in $]-\infty, a_0[, ]a_0,a_1[, ]a_1,a_2[, \ldots$. Quindi nei punti $a_0,a_1,a_2,\ldots$ ammette derivata destra e sinistra. 
Siccome questi punti non possono essere angolosi, $f_{|]-\infty, a_0[}$, $f_{|]a_0,a_1[}$, etc. hanno lo stesso coefficiente angolare, quindi, sfruttando la cardinalità, $f_{|]-\infty, a_0[}$
è lineare. Siccome $f_{|]-\infty, a_0[}$ è lineare, usando nuovamente l'assenza di punti angolosi abbiamo la tesi.
\end{proof}

\begin{examplebb}
L'insieme degli elementi di una successione crescente di successioni crescenti è un insieme di unicità.
\end{examplebb}

\begin{center}
	\begin{figure}[h]
		\centering
		\includegraphics[width=12.5cm]{immagini/succunic2.png}
	\end{figure}
\end{center}

Dimostriamo che $X$ è di unicità.
\begin{proof}
In ciascuno degli intervalli $]a_{i0}, a_{(i+1)0}[$, $f$ è lineare, ragionando come nell'esempio precedente, ci siamo ridotti alla situazione
- di nuovo - dell'esempio precedente con $a_i^{\prime} = a_{i0}$.
\end{proof}

\subsection{Digressione: insiemi numerabili}
\begin{definition}
	Un insieme $X$ è \vocab{numerabile} se è il supporto di una successione, $X = \{a_0,a_1,a_2,\ldots\} = \{a_i | i \in \NN\}$, con $a_i \ne a_j$ per ogni $i \ne j$.\footnote{O in altre parole se esiste $f : \NN \longrightarrow X$ biunivoca.}
\end{definition}

\begin{example}
	Alcuni esempi di insiemi numerabili sono:
	\begin{itemize}
		\item $\NN$, l'insieme dei numeri naturali, infatti, la successione $a_i = i$ realizza la bigezione.
		\item I numeri dispari, con la bigezione data da $a_i = 2i + 1$.
		\item I numeri primi, $a_i = p_i$, con $p_i$ $i$-esimo numero primo.
		\item $\ZZ$ l'insieme dei numeri interi, con la bigezione data da $a_i = \displaystyle (-1)^i \left\lceil\frac{i}{2}\right\rceil$.
	\end{itemize}
\end{example}

\begin{examplem}
L'insieme $\NN \times \NN = \{(x,y) | x,y \in \NN\}$ è numerabile.
\end{examplem}

\begin{proof}
La funzione $f : \NN \times \NN \longrightarrow \NN : (x,y) \longmapsto 2^x(1+2y) - 1$ è biunivoca (perché?), quindi $a_i = f^{-1}(i)$ enumera $\NN \times \NN$.
\end{proof}

\begin{proposition}
Un sottoinsieme infinito di un insieme numerabile è, a sua volta, numerabile.
\end{proposition}

\begin{proof}
Sia $Y \subseteq X$ con $Y$ infinito e $X = \{a_i | i \in \NN\}$. La sottosuccessione $b_j = a_{i_j}$ degli $a_*$ che appartengono a $Y$ enumera $Y$. A essere precisi 
bisognerebbe dire esattamente chi sono gli indici $i_j$. Per ricorsione:
\[ i_0 = \min\{i | a_i \in Y\} \qquad i_{j+1} = \min\{i > i_j | a_i \in Y\}
	\]
dove i minimi esistono perché $Y$ non è finito.
\end{proof}

\begin{proposition}
Se $X$ e $Y$ sono numerabili $X \times Y = \{(a,b) | a \in X, b \in Y\}$ è anch'esso numerabile.
\end{proposition}

\begin{proof}
Fissiamo $X = \{a_i | i \in \NN\}$, $Y = \{b_j | j \in \NN\}$. Siccome $\NN \times \NN$ è numerabile, $\NN \times \NN = \{(i_t,j_t)|t \in \NN\}$.
Quindi $X \times Y = \{(a_{i_t}, a_{j_t}) | t \in \NN\}$.
\end{proof}

\begin{example}
$\QQ$ è numerabile.
\end{example}

\begin{proof}
$\QQ$ è in corrispondenza biunivoca con:
\[F = \{(\text{num.},\text{den.}) | \text{num. $\in \ZZ$} \wedge \text{den. $\in\NN_{>0}$} \wedge \text{M.C.D.(num.,den.) = 1}\} \subseteq \ZZ \times \NN\]
\end{proof}

\begin{notexample}
$\RR$ non è numerabile.
\end{notexample}

\begin{proof}
Supponendo, per assurdo, che $\RR = \{a_i | i \in \NN\}$, cerchiamo un $x \in \RR$ che non compare fra gli $a_i$. Allo scopo, costruiamo la sottosuccessione $a_{i_j}$
definita per ricorrenza da:
\[ i_0 = 0 \qquad i_1 = \min\{i | a_i > a_0\} \qquad i_{j+1} = \min\{i | \, \text{$a_i$ è compreso tra $a_{j-1}$ e $a_j$}\}
	\]
graficamente:

\begin{center}
	\begin{figure}[h]
		\centering
		\includegraphics[width=10.5cm]{immagini/RRnum.png}
	\end{figure}
\end{center}

Si vede facilmente (esercizio!) che la successione $\{a_{i_{2k}}\}_k$ è crescente, $\{a_{i_{2k+1}}\}_k$ è decrescente 
e $\displaystyle \lim_{k \to +\infty} a_{i_{2k}} \leq \lim_{k \to +\infty a_{i_{2k+1}}}$. Fissiamo $x$ tale che $\displaystyle \lim_{k \to +\infty} a_{i_{2k}} \leq x \leq \lim_{k \to +\infty} a_{i_{2k+1}}$.
Chiaramente $x$ non è nessuno degli $a_{i_j}$, perché $a_{i_2k} < x < a_{i_{2k+1}}$. Supponiamo $x = a_n$, allora ci sarà $j$ tale che $i_j < n < i_{j+1}$, ma 
questo è assurdo perché allora $x = a_n$ è compreso fra $a_{i_{j-1}}$ e $a_{i_j}$, però $n < i_{j+1}$ contro la minimalità di quest'ultimo.

\begin{exercise}
Completare la dimostrazione nel caso $n < i$.
\end{exercise}

\begin{exercise}
Dimostrare che l'insieme di Cantor $C$ non è numerabile.
\end{exercise}
\end{proof}

\pagebreak
\subsection{Tornando agli insiemi di unicità}

\begin{theorem}
[Cantor-Lebesgue]
\label{CL}
Se $X \subseteq \RR$ è chiuso e numerabile, allora $X$ soddisfa il \hyperref[unicità]{Fatto 1.5}, ed è, quindi, di unicità.
\end{theorem}

La strategia di dimostrazione passa attraverso una definizione.

\begin{definition}
Dato $X \subseteq \RR$, il \vocab{derivato di Cantor-Bendixson} di $X$ è:
\[ X^{\prime} = X \setminus\{\text{punti isolati di $X$}\}
	\]
(dove $a \in X$ è un \vocab{punto di accumulazione} se $\exists \varepsilon > 0 : ]a - \varepsilon, a + \varepsilon[ \cap X = \{a\}$).
\end{definition}

\begin{remark}
Se $X$ è chiuso e per $X^{\prime}$ vale il \hyperref[unicità]{Fatto 1.5}, allora anche per $X$ vale il \hyperref[unicità]{Fatto 1.5}.
\end{remark}

\begin{proof}
Occorre dimostrare che se $f$ è continua, lineare, ristretta agli intervalli aperti che non intersecano $X$, e non ha punti angolosi, allora $f$ è
lineare ristretta agli intervalli aperti che non intersecano $X^{\prime}$. Fatto questo, usando l'ipotesi su $X^{\prime}$, $f$ è lineare - abbiamo quindi
mostrato che per $X$ vale \hyperref[unicità]{Fatto 1.5}.\\
Sia $]a,b[ \cap X^{\prime} = \emptyset$, dobbiamo dire che $f_{|]a,b[}$ è lineare. Ci basta dire che per ogni $\varepsilon > 0$, $f_{|[a+\varepsilon, b-\varepsilon]}$ è lineare.
Siccome $]a,b[ \cap X^{\prime} = \emptyset$, $]a,b[ \cap X = \{\text{punti isolati di $X$}\}$. Quindi $[a+\varepsilon, b-\varepsilon] \cap X$ è finito - se così non fosse, avrebbe un punto di accumulazione 
$\alpha$ che non può essere un punto isolato di $X$ (altrimenti si avrebbe un assurdo). Per cui $f_{|[a+\varepsilon, b-\varepsilon]}$ è lineare a tratti, e, siccome non ha punti angolosi, è lineare.
\end{proof}

\begin{corollary}
Sia $X^{(n)} = X^{\prime\prime\ldots\footnote{$n$ volte.}}$. Se $X^{(n)} = \emptyset$ per qualche $n \in \NN$, allora per $X$ vale il \hyperref[unicità]{Fatto 1.5}.
\end{corollary}

\begin{proof}
Induzione su $n$.
\end{proof}

Il guaio è che ci sono chiusi numerabili per cui $X^{(n)} \ne \emptyset$, qualunque sia $n$.

\begin{example}
Vogliamo costruire $X$ chiuso e numerabile tale che $X^{(n)} \ne \emptyset$ per ogni $n \in \NN$. Cominciamo col rivedere alcuni esempi già visti.
\end{example}

\begin{center}
	\begin{figure}[h]
		\centering
		\includegraphics[width=14.5cm]{immagini/es1.png}
	\end{figure}
\end{center}

Tutti i punti sono isolati, $X^{\prime} = \emptyset$.

\pagebreak

\begin{center}
	\begin{figure}[h]
		\centering
		\includegraphics[width=14.5cm]{immagini/es2.png}
	\end{figure}
\end{center}

``Successione con punto limite". Tutti i punti sono isolati salvo $l$, quindi $X^{\prime} = \{l\}$ e $X^{\prime\prime} = \emptyset$.

\begin{center}
	\begin{figure}[h]
		\centering
		\includegraphics[width=14.5cm]{immagini/es3.png}
	\end{figure}
\end{center}

``Successione di successioni", $X^{\prime} = \{a_{10}, a_{20}, \ldots, l\}$, $X^{\prime\prime} = \{l\}$ e $X^{\prime\prime\prime} = \emptyset$.\\
Si vede che possiamo proseguire, in qualche modo, costruendo una successione di successioni di successioni, etc. $n$ volte, $X_n$. Avremo $X_n^{(n)} \ne \emptyset$, $X_n^{(n+1)} = \emptyset$. Ora costruiamo 
$X_{\omega}$ fatto così:

\begin{center}
	\begin{figure}[h]
		\centering
		\includegraphics[width=14.5cm]{immagini/esomega.png}
	\end{figure}
\end{center}

È chiaro che, per ogni $n$, $X_\omega^{(n)} \ne \emptyset$. D'altro canto, $X_\omega$ soddisfa il \hyperref[unicità]{Fatto 1.5}, perché $f$ deve essere lineare in ciascuno degli intervalli
$[a_n,a_{n+1}]$, perché $X_{n+1}$ soddisfa il \hyperref[unicità]{Fatto 1.5}, quindi ci si riduce al caso della successione.

\begin{exercise}
Perché $X_\omega$ è numerabile?
\end{exercise}

Ora potremmo pensare che, pazienza se $X_\omega$ non si smonta a furia di derivati, sarà un caso particolare. Però adesso, possiamo fare una successione di insiemi come $X_\omega$, chiamiamola $X_{\omega+1}$, e 
una successione di questi $X_{\omega+2}$, etc.\\
Al diavolo, serve un nuovo corollario!

\begin{corollary}
Se $X^{(n)}$ è di ``tipo $X_\omega$", allora per $X$ vale il \hyperref[unicità]{Fatto 1.5}.
\end{corollary}

Ok, questo corollario copre $X_\omega$, $X_{\omega + 1}$, $X_{\omega + 2}$, ma copre anche $X_{\omega \cdot 2}$?
\pagebreak
\begin{center}
	\begin{figure}[h]
		\centering
		\includegraphics[width=14.5cm]{immagini/2omega.png}
	\end{figure}
\end{center}

No: occorre un nuovo corollario.

\begin{corollary}
Se $X^{(n)}$ è di ``tipo $X_{\omega - 2}$", allora per $X$ vale il \hyperref[unicità]{Fatto 1.5}.
\end{corollary}

E poi un altro per $X_{\omega \cdot 3}$, e un altro per $X_{\omega \cdot 4}$, etc.\\
E ora abbiamo finito? No, perché possiamo costruire una nuova successione con $X_{\omega},X_{\omega \cdot 2},X_{\omega \cdot 3}$, etc.\\
Se chiamiamo questa follia $X_{\omega \cdot \omega}$, ecco che si riparte a fare successioni di $X_{\omega \cdot \omega}$. Ora si sarà capito che definiremo
una serie aritmetica di queste cose, per cui potremo fare anche $\omega^\omega$, $\omega^{\omega^{\omega}}$, etc. È questa la soluzione allora?\\
No, ogni sforzo di trovare l'induzione a capo delle induzioni è vano. Se ho $X_{\omega}$, $X_{\omega^\omega}$, $X_{\omega^{\omega^{\omega}}}$, etc., allora,
ecco che faccio una successione con queste cose, la battezzo in qualche modo - ad esempio, $X_{\varepsilon_0}$ - e si riparte!\\
Per smontare ogni possibile insieme chiuso e numerabile occorre un \textbf{nuovo tipo di induzione}, l'\vocab{induzione transfinita}, che è strettamente più potente dell'induzione aritmetica.
Questa tecnica è stata sviluppata da Cantor, forse prendendo le mosse dal problema degli insiemi di unicità, e sarà uno degli argomenti centrali del corso.

\begin{exercise}[per la fine del corso]
Dimostrare il teorema di \hyperref[CL]{Cantor-Lebesgue}.
\end{exercise}

\subsection{Giochi di parole}
Descrivere un oggetto matematico non basta per crearlo. Se bastasse, si incorrerebbe in contraddizioni come queste.
\paragraph*{Paradosso di Russell}\mbox{}\\
Tipicamente le collezioni - uso questa parola perché daremo, al termine ``insieme", un senso tecnico preciso - non sono membro di se stesse: la collezione di 
tutti i numeri primi non è un numero primo. Però ci sono anche collezioni che sono membra di se stesse: per esempio la collezione di tutte le collezioni. Consideriamo:
\[ N = \{\text{collezioni $X$}\, | X \not\in X \}
	\]
la collezione delle collezioni che non sono membra di se stesse - $N$ per collezioni normali. Quindi ci chiediamo se $N \in N$ oppure no? $N \in N$ se e solo se per definizione $N \not \in N$, che è assurdo.\\
Il paradosso di Russell ci dice che, del principio di collezione - ossia l'idea che data una proprietà ben definita $P$ si possa costruire la collezione $\{X | P(X)\}$ - non ci si può fidare.

\paragraph*{Paradosso di Berry}\mbox{}\\
L'italiano annovera un numero finito di parole, è quindi possibile formare solo un numero finito di frasi di meno di centro parole. Alcune di queste descrivono un numero naturale, altre no. Comunque, solo un numero 
finito di numeri naturali può essere descritto con meno di cento parole. Per il principio del minimo, esiste:
\begin{align*}
	h = \text{``il più piccolo numero naturale che l'italiano non può} \\ 
 \text{descrivere con meno di cento parole"}
\end{align*}
Il guaio chiaramente, è che lo abbiamo appena descritto con sedici parole.\\
Quindi non ci si può fidare troppo neppure dell'italiano, o meglio, non è possibile descrivere precisamente cosa sia una descrizione precisa.\\
In conclusione, occorre fissare un linguaggio formale in cui si esprimano le proposizioni della teoria degli insiemi, e occorre fissare un sistema di assiomi, espressi in questo linguaggio, che 
dicano quali costruzioni sono lecite: quali insiemi esistono. Il ruolo della teoria degli insiemi è, poi, di fondare l'edificio della matematica. L'ambizione, quindi, è che il linguaggio e gli assiomi della teoria degli insiemi, 
siano in realtà, il linguaggio e gli assiomi della matematica.

\subsection{Scopi del corso}
Questo corso persegue due obiettivi:
\begin{enumerate}[(1)]
	\item Studiare i \textbf{fondamenti della matematica}, nella forma più comunemente accettata nel XX secolo e fino ad ora, la teoria degli insiemi di 
	\href{https://it.wikipedia.org/wiki/Ernst_Zermelo}{\textcolor{purple}{Zermelo}}-\href{https://it.wikipedia.org/wiki/Adolf_Abraham_Halevi_Fraenkel}{\textcolor{purple}{Fraenkel}} con l'assioma della scelta (ZFC).
	\item Studiare tecniche e strumenti che sono stati sviluppati grazie alla teoria degli insiemi, per esempio: la teoria delle cardinalità, la teoria dei numeri ordinali, l'induzione e la ricorsione transfinita.
\end{enumerate}

In questo corso non ci occupiamo dei modelli della teoria degli insiemi. Mi spiego. Per esempio, in teoria dei gruppi si assiomatizza cosa sia un gruppo, e poi si studia come possano essere fatti i diversi gruppi. In 
teoria degli insiemi si assiomatizza l'universo di tutti gli insiemi, però, per il teorema di incompletezza di \href{https://it.wikipedia.org/wiki/Kurt_G%C3%B6del}{\textcolor{purple}{Gödel}}, questa assiomatizzazione non 
può essere completa. Quindi esistono tanti universi insiemistici possibili. Indagare queste possibilità - i modelli della teoria degli insiemi - è argomento di corsi più avanzati.

\newpage
\section{Il linguaggio della teoria degli insiemi}
Per non incorrere in contraddizione, accettiamo che le sole proposizioni ad avere senso siano quelle esprimibili mediante \vocab{formule insiemistiche}. Le formule si costruiscono ricorsivamente.
\begin{itemize}
	\item Le lettere $a,b,c,\ldots,A,B,C,\ldots,\alpha,\beta,\gamma,\ldots$ rappresentano \vocab{variabili}. I valori delle variabili sono sempre insiemi, e non ci sono altri oggetti salvo gli insiemi.
	\item Le \vocab{formule atomiche} sono:
	\[ \text{variabile = variabile} \qquad \qquad \text{variabile $\in$ variabile}\footnote{\,``appartiene a".}
		\]
	sono formule atomiche $x=y$, $x=x$, $\alpha = C$, e anche $x \in y$, $x \in x$, $\alpha \in C$.
	\item Le formule atomiche si combinano tra loro mediante:
	\begin{itemize}
		\item \vocab{connettivi logici} ovvero il ``non'' la ``e'' e la ``o'' (inclusiva):
		\[ \text{$\neg$ formula} \qquad \text{formula $\land$ formula} \qquad \text{formula $\lor$ formula}
			\]
		quindi ad esempio:
		\begin{flalign*}
			&\neg\Phi \equiv \text{``$\Phi$ è falsa''} &\\
			&\Phi \land \psi \equiv \text{``$\Phi$ e $\psi$ sono entrambe vere''} &\\
			&\Phi \lor \psi \equiv \text{``almeno una fra $\Phi$ e $\psi$ è vera''}
		\end{flalign*}
		\item \vocab{quantificatori} ovvero quello universale ``per ogni'' e quello esistenziale ``esiste'':
		\[ \forall x \, \text{formula} \qquad \exists x \, \text{formula}
			\]
		ad esempio:
		\begin{flalign*}
			&\forall x \, \Phi \equiv \text{``$\Phi$ è vera qualunque sia l'insieme $x$''} &\\
			&\exists x \, \Phi \equiv \text{``c'è un insieme $x$ che fa si che $\Phi$ sia vera''}
		\end{flalign*}
		\begin{exercise}
			Chiaramente varranno $\forall x \, x = x,$ $ \forall x \, \exists y \, x = y,$ $ \neg \exists x \, \forall y \, x = y$.
		\end{exercise}
	\end{itemize}
\end{itemize}

\textbf{\underline{L'intuizione}} è che l'universo insiemistico sia un gigantesco grafo diretto (aciclico) i cui vertici sono gli insiemi,
ed in cui le frecce rappresentano la relazione di appartenenza.

\begin{center}
	\begin{figure}[h]
		\centering
		\includegraphics[width=10.5cm]{immagini/graf.png}
	\end{figure}
\end{center}

Possiamo solo fare affermazioni a proposito di vertici e frecce di questo grafo. Per esempio:
\[ \text{``$a$ è un elemento di un certo $b$''} \equiv \text{``c'è un percorso di due frecce fra $a$ e $b$''} 
	\]
che corrisponde mediante formule insiemistiche a $ \exists x \, a \in x \land x \in b$. E ancora:
\[\text{``$a$ è un sottoinsieme di $b$''} \equiv \text{``ogni elemento di $a$ è elemento di $b$''} \equiv \]\[
		\equiv\text{``non c'è un insieme che è elemento di $a$ e non di $b$''}\equiv\]\[
	 \equiv \text{``non c'è un vertice con una freccia verso $a$ e non una verso $b$''}
	\]
che corrisponde mediante formule insiemistiche a $\neg\exists x \, x \in a \land \neg x \in b$ (tutto ciò che raggiunge $a$ deve raggiungere anche $b$).\\
\textbf{\underline{Parentesi}} Ad essere precisi, avremmo dovuto definire le formule includendo un mucchio di parentesi, allo scopo di eliminare ogni possibilità
di formare una combinazione di simboli ambigua. Per esempio $\textcolor{red}{\Phi_1 \land \Phi_2 \lor \Phi_3}$ è ambigua, perché si potrebbe leggere $(\Phi_1 \land \Phi_2) \lor \Phi_3$
o $\Phi_1 \land (\Phi_2 \lor \Phi_3)$. In una notazione completamente parentesizzata, per esempio, la formula per ``$a$ è un sottoinsieme di $b$'' sarebbe:
\[ \neg(\exists x((x \in a)\land(\neg(x \in b))))
	\]
Non useremo, in generale, questa notazione, ma useremo le parentesi selettivamente per evitare ambiguità.\\
\textbf{\underline{Aberrazioni}} Le formule appena descritte costituiscono il linguaggio della teoria degli insiemi \textbf{puro}. Durante il corso estenderemo
più volte questo linguaggio mediante abbreviazioni, che semplicemente rimpiazzano formule più lunghe con scritture convenzionali più compatte, e quindi non alterano 
la potenza espressiva del linguaggio. Vediamo le prime abbreviazioni:
\[ x \ne y \Mydef \neg x = y \qquad x \not\in y \Mydef \neg x \in y \qquad \not\exists \Phi\footnote{DA COMPLETARE}
	\] 
\begin{note}
	Il fatto che possiamo dire $C = A \cup B$ o $C = A \cap B$ non significa né che questi oggetti esistano né che siano unici. Dimostreremo fra poco l'esistenza e unicità 
	di unione e intersezione.
\end{note}

\begin{exercise}
Esprimi queste proposizioni mediante formule insiemistiche pure:
\begin{itemize}
	\item gli elementi degli elementi di $A$ sono elementi di $A$;
	\item $B$ è l'insieme dei sottoinsiemi di $A$;
	\item l'unione degli elementi di $A$ è l'intersezione di quelli di $B$\footnote{Qui assumi che l'unione e intersezione.}
\end{itemize}
\end{exercise}

\subsection{Le regole di inferenza}
La teoria assiomatica degli insiemi si compone di tre parti: il linguaggio formale che abbiamo appena descritto, gli assiomi della teoria che studieremo durante il corso, 
ed un sistema di regole che specificano precisamente quali passaggi sono leciti nelle dimostrazioni. Possiamo immaginare questa ultima componente come una specie di algebra dei ragionamenti,
che permette di verificare i passaggi di una dimostrazione in maniera puramente meccanica, come se fossero semplici manipolazioni algebrica. Noi non vedremo le regole di inferenza, e voglio spiegare qui il perché.
\begin{enumerate}[1]
	\item Sono argomento del corso di logica.
	\item In realtà, scrivere le dimostrazioni in maniera formale, le renderebbe lunghissime e particolarmente incomprensibili.
	\item In pratica, non si sbaglia facendo ragionamenti che non reggono, si sbaglia dicendo cose fumose che non possono essere espresse nel linguaggio della teoria. Per esempio, le parole ``e così via'' sono pericolose.
	\item Conoscere le regole - fidatevi - non aiuta né a trovare né a capire le dimostrazioni.
\end{enumerate}
Pur senza dare un sistema completo di regole, vediamo qualche manipolazione formale che potrebbe servire.\\
\textbf{\underline{Tavole di verità}} Due combinazioni mediante connettivi logici ($\neg$, $\land$, $\lor$, $\rightarrow$, $\leftrightarrow$)
delle stesse formule - ``\vocab{combinazioni booleane}'' - alle volte, dicono la stessa cosa. Per esempio, $\neg \Phi \lor \neg \psi \equiv\footnote{\,``equivale a''.} \neg (\Phi \land \psi)$.
Per verificare questo fatto basta considerare tutte le possibili combinazioni di valori di verità che possono assumere le formule combinate - nell'esempio $\Phi$ e $\psi$ - compilando una ``\vocab{tabella di verità}''.
\begin{center}
	\begin{tabular}{>{$}l<{$}>{$}l<{$}|*{7}{>{$}l<{$}}}
	\Phi & \psi & \neg\Phi   & \neg\psi   & \neg\Phi \lor \neg\psi   & \Phi \land \psi & \neg(\Phi \land \psi)    \\
	\hline\vrule height 14pt width 0pt
	V & V & F & F & \textcolor{red}{F} & V & \textcolor{red}{F}\\
	V & F & F & V & \textcolor{red}{V} & F & \textcolor{red}{V}\\
	F & V & V & F & \textcolor{red}{V} & F & \textcolor{red}{V}\\
	F & F & V & V & \textcolor{red}{V} & F & \textcolor{red}{V}
	\end{tabular} 
\end{center}
Come si osserva le due colonne corrispondenti ai valori di verità delle nostre formule iniziali hanno gli stessi valori di verità in ogni caso.\\
Conviene tenere a mente alcune delle equivalenze elementari:
\[ \neg\neg \Phi \equiv \Phi \qquad \Phi \land (\psi \lor \Theta) \equiv (\Phi \land \psi) \lor (\Phi \land \Theta) \qquad \Phi \lor (\psi \land \Theta) \equiv (\Phi \lor \psi) \land (\Phi \lor \Theta)
	\]\[ \neg(\Phi \land \psi) \equiv \neg \Phi \lor \neg \psi \qquad \neg(\Phi \lor \psi) = \neg \Phi \land \neg \psi
		\]\[ \Phi \rightarrow \neg \psi \equiv \psi \rightarrow \neg \Phi \qquad \Phi \rightarrow \psi \equiv \neg \psi \rightarrow \neg \Phi
			\]

\begin{exercise}
Dimostrare le equivalenze delle formule elencate sopra.
\end{exercise}

Per quanto riguarda i quantificatori ricordiamo le regole seguenti, che tuttavia non sono esaustive.
\[ \neg\forall x \, \Phi \equiv \exists x \, \neg\Phi \qquad \neg\forall x \, \neg \Phi \equiv \exists x \, \Phi
	\]\[ \neg\exists x \, \Phi \equiv \forall x \, \neg \Phi \qquad \neg \exists x \, \neg \Phi \equiv \forall x \, \Phi
		\]

\begin{exercise}
Convinciti della validità delle equivalenze precedenti.
\end{exercise}

\begin{exercise}
Dimostra che:
\[ \neg \forall x \in A \, \Phi \equiv \exists x \in A \, \neg \Phi \qquad \neg \exists x \in A \, \Phi \equiv \forall x \in A \, \Phi
	\]
\end{exercise}

\begin{exercise}
Dimostra che:
\[ \forall x (x \in A \rightarrow x \in B) \equiv \neg \exists x (x \in A \land \neg x \in B)
	\]
\end{exercise}

\begin{exercise}
Secondo te, la seguente formula è vera?
\[ \forall A ((\exists x \, x \in A) \rightarrow \exists x \in A (x \in B \rightarrow \forall y \in A \, y \in B))
	\]
\end{exercise}

Infine vi sono regole per la relazione di uguaglianza, che dicono, in sostanza, che se $x = y$ allora $x$ e $y$ non sono distinguibili, ossia vale $\Phi(x) \leftrightarrow \Phi(y)$ qualunque sia $\Phi$.
Per quanto ci riguarda, \textbf{se $x = y$ allora $x$ e $y$ sono nomi della stessa cosa}.

\newpage
\section{I primi assiomi}
\subsection{Assiomi dell'insieme vuoto e di estensionalità}
\begin{axiom}
[Assioma dell'insieme vuoto]
Esiste un insieme vuoto.
\[ \exists x \; \forall y \; y \not\in x
		\]
\end{axiom}

\begin{note}
Questo assioma non sarebbe strettamente necessario, in quanto potremmo ottenere un insieme vuoto anche come sottoprodotto, per esempio, dell'assioma dell'infinito che vedremo in seguito.
Tuttavia è bello poter partire avendo per le mani almeno un insieme.
\end{note}

\begin{axiom}
[Assioma di estensionalità]
\label{estensionalità}
Un insieme è determinato dalla collezione dei suoi elementi. Due insiemi coincidono se e solo se hanno i medesimi elementi.
\[ \forall a \; \forall b \; a = b \leftrightarrow \forall x (x \in a \longleftrightarrow x \in b)
	\]
\end{axiom}

\begin{exercise}
Dimostra che la freccia $a = b \rightarrow \forall x (x \in a \longleftrightarrow x \in b)$, in realtà, segue dal fatto che se $a = b$ allora $a$ e $b$ sono indistinguibili.
\end{exercise}

\textbf{\underline{Convenzione}} Le variabili libere (= non quantificate), se non specificato altrimenti, si intendono quantificate universalmente all'inizio della formula. Per cui possiamo scrivere
l'assioma di estensionalità semplicemente nella forma:
\[ a = b \leftrightarrow \forall x (x \in A \leftrightarrow x \in b)
	\]

\begin{proposition}
C'è un unico insieme vuoto.
\[ \exists ! \; x \; \forall y \; y \not \in x
	\]
\end{proposition}

\begin{proof}
Consideriamo due insiemi vuoti $x_1$ e $x_2$, ossia supponiamo $\forall y \, y \not\in x_1$, e $\forall y \, y \not \in x_2$. Allora:
\[ \forall y \, y \in x_1 \leftrightarrow y \in x_2
	\]
perché $y \in x_1$ e $y \in x_2$ sono entrambe necessariamente false (dunque è sempre contemporaneamente vera la loro negazione). Quindi, per \hyperref[estensionalità]{estensionalità}, $x_1 = x_2$.
\end{proof}


\newpage
\subsection{Assioma di separazione}
\newpage
\subsection{Classi e classi proprie}
\newpage
\subsection{Assioma del paio e coppia di Kuratowski}
\newpage
\subsection{Assioma dell'unione e operazioni booleane}
\newpage
\subsection{Assioma delle parti e prodotto cartesiano}
\newpage
\subsection{Relazioni di equivalenza e di ordine, funzioni}



\newpage
\section{Assioma dell'infinito e numeri naturali}
\subsection{Gli assiomi di Peano}
\newpage
\subsection{L'ordine di omega}
\newpage
\subsection{Induzione forte e principio del minimo}
\newpage
\subsection{Ricorsione numerabile}




\newpage
\section{Cardinalità}
\subsection{Teorema di Cantor-Berstein}
\newpage
\subsection{Teorema di Cantor}
\newpage
\subsection{Operazioni fra cardinalità}





\newpage
\section{Cardinalità finite}
\subsection{Principio dei cassetti}
\newpage
\subsection{Operazioni fra le cardinalità finite}




\newpage
\section{La cardinalità numerabile}
\subsection{Insiemi numerabili in pratica}
\newpage
\subsection{Prodotto di numerabili è numerabile}
\newpage
\subsection{Numeri interi e razionali}
\newpage
\subsection{Ordini densi numerabili}
\newpage
\subsection{Il grafo random}




\newpage
\section{I numeri reali e la cardinalità del continuo}
\subsection{Caratterizzazione dei reali come ordine}
\newpage
\subsection{La cardinalità del continuo è 2 alla alef-zero}
\newpage
\subsection{Operazioni che coinvolgono la cardinalità del continuo}
\newpage
\subsection{Sottrarre un numerabile dal continuo}




\newpage
\section{Stato del corso}




\newpage
\section{I buoni ordinamenti}
\subsection{Operazioni fra buoni ordinamenti}
\newpage
\subsection{Gli ordinali di Von Neumann}
\newpage
\subsection{Assioma del rimpiazzamento}
\newpage
\subsection{Induzione e ricorsione transfinita}
\newpage
\subsection{Operazioni fra gli ordinali}




\newpage
\section{Aritmetica ordinale e forma normale di Cantor}
\subsection{Sottrazione e divisione euclidea}
\newpage
\subsection{La forma normale di Cantor}
\newpage
\subsection{Punti fissi e epsilon-numbers}
\newpage
\subsection{Operazioni in forma normale di Cantor}
\newpage





\newpage
\section{Gli alef}
\subsection{Teorema di Hartogs}
\newpage
\subsection{Somme e prodotti di alef}



\newpage
\section{L'assioma della scelta}
\subsection{Buon ordinamento implica AC}
\newpage
\subsection{AC implica buon ordinamento (idea)}
\newpage
\subsection{Zorn implica buon ordinamento}
\newpage
\subsection{AC implica Zorn}
\newpage
\subsection{Conseguenze immediate di AC}
\newpage
\subsection{Esempi di applicazione di AC}
\newpage
\subsection{Basi di spazi vettoriali}
\newpage
\subsection{Invariante di Dehn}
\newpage
\subsection{Insieme di Vitali}
\newpage
\subsection{Teorema di Cantor-Bendixson}
\newpage
\subsection{Teorema di Tarski sulla scelta}









\newpage
\section{Aritmetica cardinale}
\subsection{Somme e prodotti infiniti}
\newpage
\subsection{Teorema di König}
\newpage
\subsection{Cofinalità}
\newpage
\subsection{Formula di Hausdorff}






\newpage
\section{Gerarchia di Von Neumann}
\subsection{Formule relativizzate ad una classe}
\newpage
\subsection{Assioma di buona fondazione}
\newpage
\subsection{Principio di epsilon-induzione}


\end{document}